
\chapter{Finite Differences\label{chap:FD}}

\stepwise{

Partial differential equations (PDEs) such as the Navier-Stokes equations
can be solved approximately by splitting up space and time into discrete
points and approximating the gradients using differences. \step{

\begin{minipage}[c]{0.48\columnwidth}%
\begin{itemize}
\item For example, if we consider only one-dimensional space (only variations
in the $x$ direction), then we can divide the space between $x=0$
and $x=1$ into $N$ equal intervals, each of size $\Delta x$, so
that $x_{j}=j\Delta x$ for $j=1,2,...,N$. 
\item We can divide time into time steps $\Delta t$, so that $t_{n}=n\Delta t,~n=0,1,2,...$.\end{itemize}
%
\end{minipage}\hfill{}%
\begin{minipage}[c]{0.45\columnwidth}%
\resizebox{1\textwidth}{!}{\input{figs/fx.pdftex_t}}%
\end{minipage}}\step{
\begin{itemize}
\item We wish to solve the 1D linear advection equation, ${\textstyle \frac{\partial\phi}{\partial t}+u\frac{\partial\phi}{\partial x}=0}$,
where $u$ is the known wind speed. \step{
\item Define $\phi_{j}^{(n)}=\phi(x_{j},t_{n})$, $\frac{\partial\phi}{\partial x}_{j}^{(n)}=\frac{\partial\phi}{\partial x}(x_{j},t_{n})$,
$\frac{\partial\phi}{\partial t}_{j}^{(n)}=\frac{\partial\phi}{\partial t}(x_{j},t_{n})$.
}\step{
\item At time $n$ and position $j$ we can make the following approximations:
\[
\frac{\partial\phi}{\partial x}_{j}^{(n)}\approx\opttext{\frac{\phi_{j+1}^{(n)}-\phi_{j-1}^{(n)}}{2\Delta x}}\ \ \ \ \ \ \frac{\partial\phi}{\partial t}_{j}^{(n)}\approx\opttext{\frac{\phi_{j}^{(n+1)}-\phi_{j}^{(n-1)}}{2\Delta t}}
\]
}\step{
\item These can be substituted into the linear advection equation to give
\[
\opttext{\frac{\phi_{j}^{(n+1)}-\phi_{j}^{(n-1)}}{2\Delta t}+u\frac{\phi_{j+1}^{(n)}-\phi_{j-1}^{(n)}}{2\Delta x}=0}
\]
}
\end{itemize}
}}

\clearpage{}

\stepwise{
\begin{itemize}
\item This can be re-arranged to give $\phi_{j}^{(n+1)}$, $\phi_{j}$ at
the next time-step as a function of $\phi$s at previous time-steps
and at adjacent locations. This can be simplified using the Courant
number, $c=u\Delta t/\Delta x$ to give:
\[
\phi_{j}^{(n+1)}=\opttext{\phi_{j}^{(n-1)}-c_{j}\bigl(\phi_{j+1}^{(n)}-\phi_{j-1}^{(n)}\bigr)}
\]
\step{
\item This is the CTCS scheme - Centred in Time, Centred in space, because
the approximations for $\partial\phi/\partial t$ and $\partial\phi/\partial x$
are both centred. }\step{
\item Python code implementing CTCS is given in section \ref{sec:CTCScode}
}
\end{itemize}
}

\clearpage{}


\section{Exercises}
\begin{enumerate}
\item Find a finite difference formula for the second derivative, $\partial^{2}\phi/\partial x^{2}$,
on a grid with spacing $\Delta x$ indexed by $j$.
\item Hence derive the forward in time, centred in space (FTCS) scheme for
the diffusion equation:
\[
\frac{\partial\phi}{\partial t}=K\frac{\partial^{2}\phi}{\partial x^{2}}.
\]

\item The equation for inertial oscillations given in section \ref{sub:Coriolis}
is:
\[
\frac{\partial\mathbf{u}}{\partial t}=\text{-}2\bm{\Omega}\times\mathbf{u}.
\]
Write equations for the horizontal components of $\mathbf{u}$, assuming
that:
\[
\mathbf{u}=\left(\begin{array}{c}
u\\
v\\
w
\end{array}\right),\ \ \ 2\bm{\Omega}=\left(\begin{array}{c}
0\\
0\\
f
\end{array}\right).
\]
Hence write down a numerical method for integrating $u$ and $v$
forward in time.
\item Derive a finite difference scheme for Burger's equation in one dimension:
\[
\frac{\partial u}{\partial t}+u\frac{\partial u}{\partial x}=0.
\]

\end{enumerate}
\optpage{


\subsection{Answers}
\begin{enumerate}
\item $\frac{\partial^{2}\phi}{\partial x^{2}}_{j}\approx\frac{\phi_{j+1}-2\phi_{j}+\phi_{j-1}}{\Delta x^{2}}$
\item $\frac{\phi_{j}^{(n+1)}-\phi_{j}^{(n)}}{\Delta t}\approx\frac{\phi_{j+1}^{(n)}-2\phi_{j}^{(n)}+\phi_{j-1}^{(n)}}{\Delta x^{2}}$
\item $\partial u/\partial t=fv$, $\partial v/\partial t=-fu$\\
$u^{(n+1)}=u^{(n)}+\Delta t\ f\ v^{(n)}$\\
$v^{(n+1)}=v^{(n)}-\Delta t\ f\ u^{(n+1)}$ 
\item $u_{j}^{(n+1)}=u_{j}^{(n)}-\Delta t\ u_{j}^{(n)}\frac{u_{j}^{(n)}-u_{j-1}^{(n)}}{\Delta x}$
\end{enumerate}
}

\clearpage{}


\section{Python Code to Solve the Linear Advection Equation using CTCS\label{sec:CTCScode}}

\lstinputlisting[basicstyle={\ttfamily\codeFont},frame=single,language=Python,tabsize=4,showstringspaces=true,basicstyle=\small]{practicals/CTCS.py}
