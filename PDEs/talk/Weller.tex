%% LyX 2.0.8.1 created this file.  For more info, see http://www.lyx.org/.
%% Do not edit unless you really know what you are doing.
\documentclass[10pt,serif]{beamer}
\usepackage[latin9]{inputenc}
\setcounter{secnumdepth}{3}
\setcounter{tocdepth}{3}
\setlength{\parskip}{0bp}
\setlength{\parindent}{0pt}
\usepackage{array}
\usepackage{mathrsfs}
\usepackage{url}
\usepackage{bm}
\usepackage{amsmath}
\usepackage{amssymb}
\usepackage{mathdots}
\usepackage{graphicx}
\PassOptionsToPackage{version=3}{mhchem}
\usepackage{mhchem}
\usepackage{esint}
\usepackage[authoryear]{natbib}

\makeatletter

%%%%%%%%%%%%%%%%%%%%%%%%%%%%%% LyX specific LaTeX commands.
%% Because html converters don't know tabularnewline
\providecommand{\tabularnewline}{\\}
%% A simple dot to overcome graphicx limitations
\newcommand{\lyxdot}{.}


%%%%%%%%%%%%%%%%%%%%%%%%%%%%%% Textclass specific LaTeX commands.
 % this default might be overridden by plain title style
 \newcommand\makebeamertitle{\frame{\maketitle}}%
 \AtBeginDocument{
   \let\origtableofcontents=\tableofcontents
   \def\tableofcontents{\@ifnextchar[{\origtableofcontents}{\gobbletableofcontents}}
   \def\gobbletableofcontents#1{\origtableofcontents}
 }
 \def\lyxframeend{} % In case there is a superfluous frame end
 \long\def\lyxframe#1{\@lyxframe#1\@lyxframestop}%
 \def\@lyxframe{\@ifnextchar<{\@@lyxframe}{\@@lyxframe<*>}}%
 \def\@@lyxframe<#1>{\@ifnextchar[{\@@@lyxframe<#1>}{\@@@lyxframe<#1>[]}}
 \def\@@@lyxframe<#1>[{\@ifnextchar<{\@@@@@lyxframe<#1>[}{\@@@@lyxframe<#1>[<*>][}}
 \def\@@@@@lyxframe<#1>[#2]{\@ifnextchar[{\@@@@lyxframe<#1>[#2]}{\@@@@lyxframe<#1>[#2][]}}
 \long\def\@@@@lyxframe<#1>[#2][#3]#4\@lyxframestop#5\lyxframeend{%
   \frame<#1>[#2][#3]{\frametitle{#4}#5}}

%%%%%%%%%%%%%%%%%%%%%%%%%%%%%% User specified LaTeX commands.
\usepackage{animate}

% hack to get natbib working with beamer
\renewcommand{\newblock}{}

% list modifications
\setlength{\leftmargini}{0em}
\setlength{\leftmarginii}{1em}

% make $\times$, $+$, $-$ and $=$ use less space
\newcommand{\tims}{\negthinspace \times \negthinspace}
\newcommand{\eq}{\negthinspace = \negthinspace}
\newcommand{\plus}{\negthinspace + \negthinspace}
\newcommand{\minus}{\text{-}}
\newcommand{\smallcdot}{\negthinspace \cdot \negthinspace}

\newcommand{\nicefrac}[2]{\ensuremath ^{#1}\!\!/\!_{#2}}
\newcommand{\half}{\nicefrac{1}{2}}
\usepackage { fancybox}

\setlength{\tabcolsep}{2pt}

\makeatother

\begin{document}

\title{\vspace{-1.5cm}
\\
Optimal Transported Meshes on the Sphere for Global Atmospheric Modelling\vspace{-0.5cm}
}


\author{Hilary Weller {\footnotesize{}(Meteorology, University of Reading)}
and \\
Phil Browne {\footnotesize{}(now at ECMWF)}}


\date{\vspace{-1cm}
\begin{tabular}{>{\centering}p{0.33\textwidth}>{\centering}p{0.33\textwidth}>{\centering}p{0.33\textwidth}}
 & 31 March 2017 & \includegraphics[scale=0.2]{/home/hilary/graphics/MetLogo}\tabularnewline
\end{tabular}}


\titlegraphic{%
\begin{tabular*}{1\textwidth}{@{\extracolsep{\fill}}l||r}
\multicolumn{2}{l}{\includegraphics[width=1\textwidth]{/home/hilary/OpenFOAM/hilary-2\lyxdot 3\lyxdot 0/run/meshes/sphereMeshes/MongeAmpereFromPpt/6/movie/pptMesh6}\vspace{-2.5cm}
}\tabularnewline
\end{tabular*}}

\makebeamertitle

\lyxframeend{}


\lyxframeend{}\lyxframe{r-Adaptivity}
\begin{itemize}
\item \textbf{\textcolor{blue}{R}}edistribution

\begin{itemize}
\item keep mesh topology fixed
\item deform the mesh based on a monitor function for more resolution where
needed
\end{itemize}
\item Why? \pause

\begin{itemize}
\item Adaptivity - more resolution only where needed \pause
\item No load balancing problems\pause
\item No need to map solution between meshes\pause
\item Retro-fit to existing codes\pause
\end{itemize}
\item Who not?

\begin{itemize}
\item Distorted meshes
\item Never have complete control of $\Delta x$, $\Delta y$ and $\Delta z$
independently\pause
\item Any more?\pause
\end{itemize}
\item First question:

\begin{itemize}
\item How do we move the mes
\end{itemize}
\end{itemize}

\lyxframeend{}


\lyxframeend{}\lyxframe{Optimally Transported Meshes in Euclidean Geometry }
\begin{itemize}
\item How to create a mesh which is equidistributed with respect to a monitor
function. ie
\[
A_{x}m\left(\mathbf{x}\right)=\text{const}
\]
 for each cell with area $A_{x}$ for mesh monitor function $m\left(\mathbf{x}\right)$.
\pause
\item Define a map from old mesh point, $\bm{\xi}$, to new mesh points,
$\mathbf{x}$ which is the gradient of a convex potential, $\phi$:
\[
\mathbf{x}=\bm{\xi}+\nabla\phi
\]
This leads to an \textbf{``optimally transported mesh''}, free of
tangling\pause
\item Jacobian determinant of mesh transform is the change in area:
\[
\det\left(\nabla\mathbf{x}\right)=A_{x}/A_{\xi}
\]

\item \pause Combine to form a Monge-Amp\'ere equation for mesh potential,
$\phi$:
\begin{eqnarray*}
\det\left(\nabla\left(\bm{\xi}+\nabla\phi\right)\right)m\left(\mathbf{x}\right) & = & \text{const}\\
\text{or }\ \det\left(I+H\left(\phi\right)\right) & = & \frac{\text{const}}{m\left(\mathbf{x}\right)}
\end{eqnarray*}

\item \pause This works fine in Euclidean geometry ...
\end{itemize}

\lyxframeend{}


\lyxframeend{}\lyxframe{Numerical Solution of the Monge-Amp�re Equation on a Plane \textmd{\normalsize{}(test
case from \citet{BRW15})}}

\begin{tabular}{ccc}
Monitor Function, $m$ & Initial Mesh & Solution of \tabularnewline
 &  & $|I+H\left(\phi\right)|m\left(\mathbf{x}\right)=\text{const}$\tabularnewline
\includegraphics[width=0.33\textwidth]{/home/hilary/OpenFOAM/hilary-2\lyxdot 3\lyxdot 0/run/meshes/plane/BuddRusselWalsh/Fig4\lyxdot 4_V0_moreDiags/1/monitor}\textrm{\pause} & \includegraphics[width=0.33\textwidth]{/home/hilary/OpenFOAM/hilary-2\lyxdot 3\lyxdot 0/run/meshes/plane/BuddRusselWalsh/blankCaseFig4\lyxdot 4_V0/constant/meshAll}\textrm{\pause} & \includegraphics[width=0.33\textwidth]{/home/hilary/OpenFOAM/hilary-2\lyxdot 3\lyxdot 0/run/meshes/plane/BuddRusselWalsh/Fig4\lyxdot 4_V0_moreDiags/1/Phi}\tabularnewline
\end{tabular}

Finite volume discretisation in space

Fixed point (Poincar�) outer iterations


\lyxframeend{}


\lyxframeend{}\lyxframe{Initial Mesh + $\nabla\phi$ gives an equidistributed mesh}

\begin{tabular}{ccc}
Initial Mesh, $\bm{\xi}$ & $+\nabla\phi$ & =$\mathbf{x}$\tabularnewline
\includegraphics[width=0.33\textwidth]{/home/hilary/OpenFOAM/hilary-2\lyxdot 3\lyxdot 0/run/meshes/plane/BuddRusselWalsh/blankCaseFig4\lyxdot 4_V0/constant/meshAll}\textrm{\pause} & \includegraphics[width=0.33\textwidth]{/home/hilary/OpenFOAM/hilary-2\lyxdot 3\lyxdot 0/run/meshes/plane/BuddRusselWalsh/Fig4\lyxdot 4_V0_moreDiags/1/Phi}\textrm{\pause} & \includegraphics[width=0.33\textwidth]{/home/hilary/OpenFOAM/hilary-2\lyxdot 3\lyxdot 0/run/meshes/plane/BuddRusselWalsh/Fig4\lyxdot 4_V0_moreDiags/1/meshAll}\tabularnewline
\end{tabular}


\lyxframeend{}\lyxframe{But on the Surface of a Sphere}
\begin{itemize}
\item The gradient of a potential does not map to point on the sphere
\item $\det\left(\nabla\nabla\phi\right)$ does not give change in area
\item $\therefore$ cannot formulate a Monge-Amp�re equation for the mesh
potential.
\end{itemize}
Instead:\pause
\begin{itemize}
\item Use exponential maps to map from old to new mesh:
\[
\mathbf{x}=\bm{\xi}+\exp_{\xi}\nabla\phi
\]

\item \textrm{\pause}Solve the optimal transport problem more directly...:
\begin{equation}
\frac{A_{x}}{A_{\xi}}=\frac{\text{const}}{m\left(\mathbf{x}\right)}\label{eq:MAsphere}
\end{equation}

\item \textrm{\pause }Linearise (\ref{eq:MAsphere}) about zero (assuming
tangent plane):
\[
\frac{A_{x}}{A_{\xi}}\approx\det\left(I+H\left(\phi\right)\right)\approx\pause1+\nabla^{2}\phi
\]

\item \textrm{\pause }Create Poincar� iterations with non-linear terms
treated explicitly:
\[
1+\nabla^{2}\phi^{n+1}=1+\nabla^{2}\phi^{n}-\frac{A_{x}}{A_{\xi}}+\frac{\text{const}^{n}}{m\left(\mathbf{x}^{n}\right)}
\]

\end{itemize}

\lyxframeend{}\lyxframe{An Optimally Transported Mesh}

Given the monitor function on the sphere:

\includegraphics[width=1\textwidth]{/home/hilary/OpenFOAM/hilary-2\lyxdot 3\lyxdot 0/run/meshes/sphereMeshes/MongeAmpereV1MedStencil/4/8/0/monitor}


\lyxframeend{}


\lyxframeend{}\lyxframe{An Optimally Transported Mesh}

The optimally transported mesh is calculated iteratively:

\animategraphics[width=\linewidth,controls,every=2,poster=first]{3}{/home/hilary/OpenFOAM/hilary-2.3.0/run/meshes/sphereMeshes/MongeAmpereV1MedStencil/4/8anim/movie/mesh}{0}{100}


\lyxframeend{}


\lyxframeend{}\lyxframe{Precipitation as a Monitor Function}

\vspace{-10pt}


{\footnotesize{}
\[
m=\frac{p+p_{\min}}{p_{\max}+p_{min}}\ \text{where}\ p_{\min}=10^{-5}\text{kg}\text{m}^{-2}\text{s}^{-1},\ p_{\max}=8.73\times10^{-4}\text{kg}\text{m}^{-2}\text{s}^{-1}
\]
}{\footnotesize \par}

\animategraphics[width=\linewidth,controls,poster=first,loop]{1}{/home/hilary/OpenFOAM/hilary-2.3.0/run/meshes/sphereMeshes/MongeAmpereFromPpt/6/movie/pptMesh}{0}{11}

{\small{}Using daily average precipitation rate, 1-12 Oct 2012, from
the NOAA-CIRES 20th Century Reanalysis version 2 (Compo et al, 2011,
\url{http://www.esrl.noaa.gov/psd/data/gridded/data.20thC_ReanV2.html})\vfill{}
See \citet*{WBBC16} }{\small \par}


\lyxframeend{}


\lyxframeend{}\lyxframe{Solving PDEs on Moving Meshes \pause}

\begin{minipage}[t]{0.7\columnwidth}%
Apply Reynolds transport theorem:
\[
\frac{d}{dt}\int_{\mathscr{V}(t)}\rho\ dV=\int_{\mathscr{V}(t)}\frac{\partial\rho}{\partial t}\ dV+\int_{S}\rho\ \mathbf{v}\cdot\mathbf{dS}
\]
to the continuity equation:
\begin{eqnarray*}
\frac{\partial\rho}{\partial t}+\nabla\cdot\left(\mathbf{u}\rho\right) & = & 0.
\end{eqnarray*}
Discretise in time and apply Gauss's divergence theorem:
\[
\frac{\mathscr{V}^{n+1}\rho^{n+1}-\mathscr{V}^{n}\rho^{n}}{\Delta t}=-\int_{S}\rho\left(\mathbf{u}-\mathbf{v}\right)\cdot\mathbf{dS}.
\]
%
\end{minipage}\hfill{} %
\begin{minipage}[t]{0.25\columnwidth}%
\noindent \begin{center}
\textbf{Definitions}:
\par\end{center}

\begin{tabular}{>{\raggedright}p{0.3\textwidth}>{\raggedright}p{1\textwidth}}
$\rho$ & density\tabularnewline
$\mathbf{u}$ & fluid velocity\tabularnewline
$\mathscr{V}(t)$ & cell volume\tabularnewline
$S(t)$ & cell surface\tabularnewline
$\mathbf{dS}$ & outward normal\tabularnewline
$\mathbf{v}(t)$ & mesh velocity\tabularnewline
\multicolumn{2}{l}{\textrm{$\left(\mathbf{u}\!-\!\mathbf{v}\right)\cdot\mathbf{dS}$}}\tabularnewline
 & Volume flux\tabularnewline
\end{tabular}%
\end{minipage}


\lyxframeend{}


\lyxframeend{}\lyxframe{Discretisation}

\begin{minipage}[t]{0.48\columnwidth}%
\textbf{Definitions}:

\begin{tabular}{cl}
$n$ & Time-step number\tabularnewline
$\Delta t$ & Time-step\tabularnewline
$\rho^{\prime}$ & Temporary value of $\rho^{n+1}$\tabularnewline
$\phi_{r}$ & Relative face flux\tabularnewline
 & $=\left(\mathbf{u}\!-\!\mathbf{v}\right)\cdot\mathbf{dS}$\tabularnewline
\end{tabular}%
\end{minipage}\hfill{} %
\begin{minipage}[t]{0.48\columnwidth}%
\phantom{}\scalebox{0.5}[0.5]{\input{figs/twoCells.pdftex_t}}%
\end{minipage}
\begin{itemize}
\item Discretise in time using RK2:\pause 
\end{itemize}
\begin{eqnarray*}
\mathscr{V}_{f}^{n+1}\rho^{\prime} & = & \mathscr{V}_{f}^{n}\rho^{n}-\Delta t\left(\sum_{\text{faces}}\rho_{f}^{n}\phi_{r}\right)\\
\mathscr{V}_{f}^{n+1}\rho^{n+1} & = & \mathscr{V}_{f}^{n}\rho^{n}-\frac{\Delta t}{2}\left(\sum_{\text{faces}}\rho_{f}^{n}\phi_{r}+\sum_{\text{faces}}\rho_{f}^{\prime}\phi_{r}\right)
\end{eqnarray*}

\begin{itemize}
\item \pause Discretise in space using OpenFOAM's finite volume linear
upwind advection scheme.
\end{itemize}
\[
\rho_{f}=\rho_{u}+(\mathbf{x}_{f}-\mathbf{x}_{u})\cdot\nabla_{u}\rho
\]



\lyxframeend{}


\lyxframeend{}\lyxframe{Results - Solid body rotation}

Test case from \citet{LLM96}

$100\times100$ cells, 2,400 time-steps for one revolution

\begin{tabular}{>{\centering}p{0.48\textwidth}>{\centering}p{0.48\textwidth}}
Fixed Mesh & monitor $=\max\left(\frac{1}{16}+\frac{||\nabla\nabla\rho||}{3\times10^{-7}},1\right)$ \tabularnewline
\end{tabular}

\animategraphics[width=0.48\linewidth,controls,poster=first]{3}{/home/hilary/OpenFOAM/hilary-dev/AMMM/run/advection/solidBodyRotationOnPlane/uniformMesh/animategraphics/T_}{0}{12}
\animategraphics[width=0.48\linewidth,controls,poster=first]{3}{/home/hilary/OpenFOAM/hilary-dev/AMMM/run/advection/solidBodyRotationOnPlane/magGradTmonitor/animategraphics/T_}{0}{12}


\lyxframeend{}


\lyxframeend{}\lyxframe{Results - Solid body rotation}

Monitor function, monitor $=\max\left(\frac{1}{16}+\frac{||\nabla\nabla\rho||}{3\times10^{-7}},1\right)$
, is smoothed with Laplacian

\animategraphics[width=0.48\linewidth,controls,poster=first]{3}{/home/hilary/OpenFOAM/hilary-dev/AMMM/run/advection/solidBodyRotationOnPlane/magGradTmonitor/animategraphics/monitor_}{0}{12}
\animategraphics[width=0.48\linewidth,controls,poster=first]{3}{/home/hilary/OpenFOAM/hilary-dev/AMMM/run/advection/solidBodyRotationOnPlane/magGradTmonitor/animategraphics/mesh_}{0}{12}


\lyxframeend{}


\lyxframeend{}\lyxframe{Moving Meshes Over Mountains}


\lyxframeend{}\lyxframe{Advection Over a Mountain}

$100\times100$ cells, 2,400 time-steps for one revolution

Cone shaped mountain with peak height half the depth of the domain

\begin{tabular}{>{\centering}p{0.48\textwidth}>{\centering}p{0.48\textwidth}}
Fixed Mesh & monitor $=\max\left(\frac{1}{16}+\frac{||\nabla\nabla\rho||}{3\times10^{-7}},1\right)$ \tabularnewline
\end{tabular}

\animategraphics[width=0.48\linewidth,controls,poster=first]{3}{/home/hilary/OpenFOAM/hilary-dev/AMMM/run/advection/solidBodyRotationOnPlane/fixedMountain/animategraphics/T_}{0}{12}
\animategraphics[width=0.48\linewidth,controls,poster=first]{3}{/home/hilary/OpenFOAM/hilary-dev/AMMM/run/advection/solidBodyRotationOnPlane/magGradTmonitorMountainEmpty/animategraphics/T_}{0}{2}


\lyxframeend{}


\lyxframeend{}\lyxframe{Advection Over a Mountain with a Moving Mesh}

$100\times100$ cells, 2,400 time-steps for one revolution, monitor
$=\max\left(\frac{1}{16}+\frac{||\nabla\nabla\rho||}{3\times10^{-7}},1\right)$ 

Cone shaped mountain with peak height half the depth of the domain

\begin{tabular}{>{\centering}p{0.48\textwidth}>{\centering}p{0.48\textwidth}}
Mesh & Uniform tracer $\phi=\half$\tabularnewline
\end{tabular}

\animategraphics[width=0.48\linewidth,controls,poster=first]{3}{/home/hilary/OpenFOAM/hilary-dev/AMMM/run/advection/solidBodyRotationOnPlane/magGradTmonitorMountainEmpty/animategraphics/mesh_}{0}{2}
\animategraphics[width=0.48\linewidth,controls,poster=first]{3}{/home/hilary/OpenFOAM/hilary-dev/AMMM/run/advection/solidBodyRotationOnPlane/magGradTmonitorMountainEmpty/animategraphics/uniT_}{0}{2}


\lyxframeend{}


\lyxframeend{}\lyxframe{Advection Over a Mountain with a Moving Mesh}
\begin{itemize}
\item Zero relative flux over the ground
\item Mass of tracer conserved exactly ...\pause
\item Volume of the domain changes as the mesh moves over the mountain\pause


\noindent \textcolor{blue}{\Large{}Possible remedies}{\Large \par}

\item \pause Conservative mapping of old to new mesh

\begin{itemize}
\item Expensive
\item Exactly what we wanted to avoid with moving meshes
\item How to move vertices to get the desired volume\pause
\end{itemize}
\item An alternative

\begin{itemize}
\item Assume that the boundary condition is zero normal flow at the fixed
mountain surface
\item Calculate a relative flux as the mountain deforms \\
$\implies$ mass inside the domain will change in proportion to the
volume
\end{itemize}
\end{itemize}

\lyxframeend{}\lyxframe{With flow over the bottom boundary}

$100\times100$ cells, 2,400 time-steps for one revolution, monitor
$=\max\left(\frac{1}{16}+\frac{||\nabla\nabla\rho||}{3\times10^{-7}},1\right)$ 

Cone shaped mountain with peak height half the depth of the domain

Relative flux over the bottom boundary

\begin{tabular}{>{\centering}p{0.48\textwidth}>{\centering}p{0.48\textwidth}}
Gaussian tracer & Mesh\tabularnewline
\end{tabular}

\animategraphics[width=0.48\linewidth,controls,poster=first]{3}{/home/hilary/OpenFOAM/hilary-dev/AMMM/run/advection/solidBodyRotationOnPlane/magGradTmonitorMountain/animategraphics/T_}{0}{12}
\animategraphics[width=0.48\linewidth,controls,poster=first]{3}{/home/hilary/OpenFOAM/hilary-dev/AMMM/run/advection/solidBodyRotationOnPlane/magGradTmonitorMountain/animategraphics/mesh_}{0}{12}


\lyxframeend{}\lyxframe{With flow over the bottom boundary}

$100\times100$ cells, 2,400 time-steps for one revolution, monitor
$=\max\left(\frac{1}{16}+\frac{||\nabla\nabla\rho||}{3\times10^{-7}},1\right)$ 

Cone shaped mountain with peak height half the depth of the domain

Relative flux over the bottom boundary

\begin{tabular}{>{\centering}p{0.48\textwidth}>{\centering}p{0.48\textwidth}}
Uniform tracer $\phi=\half$ & \tabularnewline
\end{tabular}

\animategraphics[width=0.48\linewidth,controls,poster=first]{3}{/home/hilary/OpenFOAM/hilary-dev/AMMM/run/advection/solidBodyRotationOnPlane/magGradTmonitorMountain/animategraphics/uniT_}{0}{12}


\lyxframeend{}


\lyxframeend{}\lyxframe{Remaining Questions}
\begin{itemize}
\item Will this be stable when solving the Euler equation?
\item \pause What about conservative fluxes between bottom boundary and
domain

\begin{itemize}
\item air-sea fluxes
\end{itemize}
\item \pause What are the implications of the lack of mass conservation?
\item \pause How important is exact preservation of constants?\pause
\end{itemize}
Next - solve incompressible Euler equations


\lyxframeend{}\lyxframe{OpenFOAM A-C-grid }

\begin{minipage}[t]{0.72\columnwidth}%
Incompressible Euler equations in flux form:
\begin{eqnarray*}
\frac{\partial\mathbf{u}}{\partial t}+\nabla\cdot\left(\mathbf{u}\mathbf{u}\right) & = & -\nabla p\\
\nabla\cdot\left(\mathbf{u}\right) & = & 0
\end{eqnarray*}


\resizebox{0.3\paperwidth}{!}{\input{figs/2dgeom.pdftex_t}}\hfill{}\resizebox{0.3\paperwidth}{!}{\input{figs/2dvars.pdftex_t}}%
\end{minipage}\hfill{} %
\begin{minipage}[t]{0.25\columnwidth}%
\noindent \begin{center}
\textbf{Definitions}:
\par\end{center}

\begin{tabular}{>{\raggedright}p{0.3\textwidth}>{\raggedright}p{1\textwidth}}
p & pressure (projection)\tabularnewline
$\mathbf{u}$ & fluid velocity\tabularnewline
$\mathbf{u}_{f}$ & velocity at face\tabularnewline
$U_{S}$ & $=\mathbf{u}_{f}\cdot\mathbf{S}$ (flux)\tabularnewline
\end{tabular}%
\end{minipage}


\lyxframeend{}\lyxframe{OpenFOAM A-C-grid }
\begin{itemize}
\item Apply Reynolds transport theorem to momentum in each cell of volume
$\mathscr{V}(t)$: 
\[
\frac{\mathscr{V}^{n+1}\mathbf{u}^{n+1}-\mathscr{V}^{n}\mathbf{u}^{n}}{\Delta t}=-\int_{S}\mathbf{u}\left(\mathbf{u}-\mathbf{v}\right)\cdot\mathbf{dS}-\int_{\mathscr{V}}\nabla p\ dV
\]

\item \pause  Interpolate onto cell faces $(\ )_{f}$ and take dot product
with $\mathbf{S}$:
\begin{eqnarray*}
\frac{\mathscr{V}_{f}^{n+1}U_{S}^{n+1}-\mathscr{V}_{f}^{n}\mathbf{u}^{n}\cdot\mathbf{S}^{n+1}}{\Delta t} & = & -\mathbf{S}^{n+1}\cdot\left(\sum_{\text{faces}}U_{r}\mathbf{u}\right)_{f}^{n+\half}-\left(\mathscr{V}_{f}\nabla_{S}p\right)^{n+\half}
\end{eqnarray*}
where relative flux is $U_{r}=\left(\mathbf{u}-\mathbf{v}\right)\cdot\mathbf{S}$
, pressure gradient $\nabla_{S}p=\mathbf{S}\cdot\nabla p$ 
\end{itemize}

\lyxframeend{}\lyxframe{Interacting Vortices, incompressible Euler equations}

\begin{tabular}{>{\centering}p{0.5\linewidth}c}
$400\times400$ uniform mesh & $200\times200$ uniform mesh\tabularnewline
\animategraphics[width=0.5\linewidth,controls,poster=first]{5}{/home/hilary/OpenFOAM/hilary-dev/AMMM/run/movingIcoFoam/vorticesFplane_doublyPeriodic/hiResUniformMeshRC/animategraphics/vorticity_}{0}{40} 
&
 %\animategraphics[width=0.5\linewidth,controls,poster=first]{10}{/home/hilary/OpenFOAM/hilary-dev/AMMM/run/movingIcoFoam/vorticesFplane_doublyPeriodic/uniformMeshRC/animategraphics/vorticity_}{0}{40}\tabularnewline
\end{tabular}


\lyxframeend{}


\lyxframeend{}\lyxframe{Conclusions}
\begin{itemize}
\item Equation of Monge-Amp�re type solved on the surface of a sphere for
generating an optimally transported mesh\pause
\item Fluids equations on moving meshes on curved domains (mountains)

\begin{itemize}
\item Accuracy gains for linear advection
\item need small flow over boundary to maintain preservation of constants\\
$\rightarrow$ lack of mass conservation\pause
\end{itemize}
\item Shallow water equations on moving meshes on planar domains

\begin{itemize}
\item Solving Monge-Amp�re equation is $\sim\frac{1}{5}$ of total cost
\item Tighter time-step restrictions
\item Maintains tighter gradients
\item Inaccuracies due to distorted meshes
\end{itemize}
\end{itemize}

\lyxframeend{}\lyxframe{References}

\textcolor{black}{\tiny{}\bibliographystyle{plainnat}
\bibliography{numerics}
}{\tiny \par}


\lyxframeend{}
\end{document}
